\documentclass{article}
\usepackage[T2A]{fontenc}
\usepackage[utf8]{inputenc}

\usepackage[russian]{babel}
\usepackage{amsmath,amsfonts,amsthm,amssymb,amsbsy,amstext,amscd,amsxtra,multicol}
\usepackage{verbatim}
\usepackage{tikz}
\usetikzlibrary{automata,positioning}
\usepackage{multicol}
\usepackage{graphicx}
\usepackage[colorlinks,urlcolor=blue]{hyperref}
\usepackage[stable]{footmisc}
\usepackage{ dsfont }
\usepackage{wrapfig}
\usepackage{xparse}
\usepackage{ifthen}
\usepackage{bm}
\usepackage{color}
%\usepackage{subfigure}
\usepackage{subcaption}
\usepackage{indentfirst}
\usepackage{enumitem}


\usepackage{xcolor}
\usepackage{hyperref}
\definecolor{linkcolor}{HTML}{799B03} % цвет гиперссылок
\definecolor{urlcolor}{HTML}{799B03} % цвет гиперссылок
\usepackage[left=2cm,right=2cm,top=2cm,bottom=2cm]{geometry}
\title{HW2: выпуклая оптимизация, двойственность, условия Каруша-Куна-Таккера}
\author{Семинарист: Курузов Илья}
\date{Дедлайн: 23:59, 18.10.2021}

\DeclareMathOperator{\dist}{dist}
\DeclareMathOperator{\var}{var}
\DeclareMathOperator{\quartile}{quartile}

\begin{document}

\maketitle

\begin{enumerate}
\section{Выпуклая оптимизация}
    \item $[2]$ Рассмотрим задачу аппроксимации по  $\ell_p$ норме
    \begin{equation*}
        \min_{\mathbf{x}\in \mathbb{C}^n}\|A\mathbf{x}-\mathbf{b}\|_p
    \end{equation*}
    с комплексными параметрами $A\in\mathbb{C}^{m\times n}, \mathbf{b}\in\mathbb{C}^m$ и комплекснозначной переменной $\mathbf{x}\in\mathbb{C}^n$. Выразите задачу аппроксимации по $\ell_p$ норме для $p=1,2,\infty$ как QCQP или SOCP задачу с вещественными переменными и параметрами.
    
    Напомним, что норма $\ell_p$ определяется как 
    $\|\mathbf{x}\|_p = \left(\sum\limits_{i=1}^n |x_i|^p\right)^{\frac{1}{p}}$
    для $p\geq 1$ и $\|\mathbf{x}\|_\infty = \max_{i=\overline{1,n}}|x_i|$ как предельный случай для $p=\infty$.
    
    \item $[5]$ В данной задаче Вам предлагается привести задачу робастного квадратичного программирования к стандартной форме. Сама задача квадратичного робастного программирования формулируется следующим образом:
    \begin{equation}
    \begin{aligned}
                &\min_{\mathbf{x}\in\mathbb{R}^n} \sup_{P\in\mathcal{P}}\left(\mathbf{x}^\top P\mathbf{x} +\mathbf{b}^\top\mathbf{x}+c\right)\\
                &\text{s.t. } A\mathbf{x}\leq \mathbf{d},
    \end{aligned}
    \end{equation}
    где $A\in\mathbf{R}^{m\times n}, \mathbf{b}\in\mathbf{R}^n, \mathbf{d}\in\mathbf{R}^m$ - параетры задачи. Множество $\mathcal{P}\subseteq \mathbb{S}^n_{+}$ есть некоторое множество неотрицательно определенных матриц. Приведите задачу к стандартному виду (например, QP, QCQP, SCP, SDP) для следующих множеств $\mathcal{P}$:
    \begin{enumerate}[label=\alph*)]
        \item $[1]$ Конечное множество матриц: $\mathcal{P}=\{P_1\dots P_k\},\text{ где } P_j\in\mathbb{S}^n_+, j=\overline{1,k}$.
        
        \item $[2]$ Множество, заданное как ограничение на собственное число матрицы $P-P_0$: $$\mathcal{P}=\left\{P\in\mathbb{S}^n_+\Big|-\gamma I \preceq P - P_0 \preceq \gamma I\right\}$$
        для некоторого $\gamma \in\mathbb{R}_+$ и $P_0\in\mathbb{S}^n_+$ (матрица $I$ есть единичная матрица).
        
        \item $[2]$ Эллипсоид матриц:
        $$\mathcal{P}=\left\{P_0+\sum\limits_{i=1}^k P_i u_i\Big| \forall \mathbf{u}\in\mathbb{R}^k, \|\mathbf{u}\|_2\leq 1 \right\},$$
        где $P_i\in\mathbb{S}^n_+,i=\overline{0,k}$ есть неотрицательно определенные матрицы.
    \end{enumerate}
    
    Постарайтесь свести задачу к более узкому классу. Например, если Вы получили задачу в форме SDP и можете свести ее к QCQP, то сведите.
    
    \item  Выразите следующие проблемы в форме задачи геометрического программирования:
    \begin{enumerate}[label=\alph*)]
        \item $[1]$ $\min\limits_{\mathbf{x}\in\mathbb{R}^n_+}\exp(p(\mathbf{x}))+\exp(q(\mathbf{x}))$, где $p,q$ - позиномы.
        \item $[1]$ $\min\limits_{\mathbf{x}\in\mathbb{R}^n_+}\frac{p(\mathbf{x})}{r(\mathbf{x})-q(\mathbf{x})}$ при ограничении $r(\mathbf{x})\geq q(\mathbf{x})$, где $p$ - позином, $q,r$ - мономы.
    \end{enumerate}

\section{Двойственность}
    \item $[2]$  Выведите двойственную задачу для задачи:
    \begin{equation*}
        \min_{\mathbf{x}\in\mathbb{R}^n} \|A\mathbf{x}-\mathbf{b}\|_2^2+\frac{\lambda}{2}\|\mathbf{x}\|_2^2,
    \end{equation*}
    введя новую переменную $\mathbf{y}\in\mathbb{R}^m$ $\mathbf{y}=A\mathbf{x}-\mathbf{b}$ и соответствующие ограничения. Параметры: $A\in\mathbb{R}^{m\times n}$, $\lambda \geq 0$. Таким образом, мы получим задачу, которая дает нижнюю оценку на решение для задачи безусловной минимизации.
    \item $[3]$ Рассмотрим задачу бинарного линейного программирования:
    \begin{equation*}
        \begin{aligned}
      \min_{x\in\mathbb{R}^n}\textbf{c}^\top \textbf{x} \\
        \text{s.t. } A\textbf{x}\leq \textbf{b}\\
        x_i(1-x_i) = 0, i=\overline{1,n}     
        \end{aligned}
    \end{equation*}
   с непустым допустимым множеством.
    \begin{enumerate}[label=\alph*)]
        \item $[1]$ Постройте двойственную задачу. Данная двойственная задача будет давать оценку снизу на оптимальное значение исходной дискретной задачи. Такой метод релаксации называется релаксацией Лагранжиана.
        
        \item $[1]$ Упростите полученную в предыдущем пункте задачу, аналитически решив ее по одному из блоков переменных. 
        
        \item $[1]$ Докажите, что оценка, которую дает двойственная задача, совпадает с решением релаксации этой задачи:
        \begin{gather*}
            \textbf{c}^\top \textbf{x}\rightarrow \min_{x\in\mathbb{R}^n}\\
            \text{s.t. } A\textbf{x}\leq \textbf{b}\\
            0\leq x_i\leq 1, i=\overline{1,n}.
        \end{gather*}     
        Hint: Один из возможных способов сделать это - вывести двойственную задачу к этой релаксации и показать эквивалентность этой двойственной задачи и полученной в пункте b.
    \end{enumerate}
    
    \item $[2]$  Рассмотрим следующую модификации задачи Optimal Experiment Design:
        \begin{equation*}
        \begin{aligned}
      \min_{\mathbf{x}\in\mathbb{R}^p} \text{tr}\left(\sum\limits_{i=1}^p x_i \mathbf{v}_i \mathbf{v}_i^\top\right)^{-1} \\
        \text{s.t. } \mathbf{x}\geq\mathbf{0}, \quad \mathbf{1}^\top \mathbf{x}=1.
        \end{aligned}
    \end{equation*}
    Данная задача называется A-optimal Design. Область определения функции $\{\mathbf{x}\in\mathbb{R}^p |\sum\limits_{i=1}^n x_i \mathbf{v}_i \mathbf{v}_i^\top\in\mathbb{S}^n_{++}\}$. Параметрами данной задачи являются $p$ вектором $\mathbf{v}_1\dots \mathbf{v}_p\in\mathbb{R}^n$. Введите новую переменную $X\in\mathbb{S}^n$ и ограничение $X = \sum\limits_{i=1}^n x_i \mathbf{v}_i \mathbf{v}_i^\top$. Выведите для новой проблемы двойственную задачу. Упростите её, насколько сможете.
    
            Hint: Можно пользоваться без доказательства, что $\nabla_X\text{tr}X^{-1}=-X^{-2}$ для $X\in\mathbb{S}^n$.
\section{Условия ККТ}

\item   \begin{enumerate}[label=\alph*)]
        \item $[2]$ Пусть параметры $\mathbf{a},\mathbf{b}\in\mathbb{R}^n_{++}$ имеют положительные компоненты,  при этом компоненты первого вектора отсортированы в порядке убывания $a_n\geq a_k \geq \dots a_1 > 0$, а компоненты второго вектора определены, как $b_k=\frac{1}{a_k}$. Выведите условия ККТ для задачи
        \begin{equation*}
        \begin{aligned}
      \min_{\mathbf{x}\in\mathbb{R}^n} -\log(\mathbf{a}^\top \mathbf{x})-\log(\mathbf{b}^\top \mathbf{x})\\
              \text{s.t. } \mathbf{x}\geq 0,\quad \mathbf{1}^\top \mathbf{x}=1
        \end{aligned}
    \end{equation*}  
    и покажите, что вектор $\mathbf{x}=\left(\frac{1}{2}, 0, 0\dots, 0, 0, \frac{1}{2}\right)^\top$ является решением этой задачи.
        \item $[1]$ Пусть $A\in\mathbb{S}^n_{++}$. Примените результат первой части задачи для вектора $\mathbf{a}=(\lambda_n\dots \lambda_1)^\top$, собственные значения в котором расположены в порядке убывания, чтобы доказать неравенство Канторовича:
        $$2\left(\mathbf{u}^\top A\mathbf{u}\right)^{\frac{1}{2}}\left(\mathbf{u}^\top A^{-1}\mathbf{u}\right)^{\frac{1}{2}}\leq \sqrt{\frac{\lambda_{n}}{\lambda_1}}+\sqrt{\frac{\lambda_{1}}{\lambda_n}},$$
        где вектор $\mathbf{u}\in\mathbb{R}^n, \|\mathbf{u}\|_2=1$.
        
        Hint: Если $A \mathbf{v}=\lambda \mathbf{v}$ и $A$ обратимая матрица, то $A^{-1}\mathbf{v} = \frac{1}{\lambda} \mathbf{v}$.

    \end{enumerate}     
        

    \item $[2]$ Выпишите условия ККТ для следующей задачи:
    \begin{gather*}
        \|A\textbf{x}-\textbf{b}\|_2^2\rightarrow \min_{x\in\mathbb{R}^n}\\
        \text{s.t. } G\textbf{x}= \textbf{h},\\
    \end{gather*}
    $A\in\mathbb{R}^{m \times n},\text{rank } A = n, G \in \mathbb{R}^{p \times n},\text{rank } G = p $ и найдите выражения для решения прямой задачи $\textbf{x}^*$ и двойственной $\nu^*$ (одна из этих точек может выражаться через другую).

    \item $[4*]$ Дивергенцией Кульбака-Лейбера между двумя распределениями $\mathcal{P}_1$ и $\mathcal{P}_2$ с абсолютно непрерывными распределениями и функциями распределения $p_1$ и $p_2$ называется величина:
    \begin{equation*}
        D_{KL}(\mathcal{P}_1||\mathcal{P}_2)=\mathbb{E}_{x\sim\mathcal{P}_1}\left[\log\frac{p_1(x)}{p_2(x)}\right]=\int\limits_{x\in X}p_1(x)\log\frac{p_1(x)}{p_2(x)}dx.
    \end{equation*}
    
    \begin{enumerate}[label=\alph*)]
     \item $[1]$ Покажите, что если распределения $\mathcal{P}_1=\mathcal{N}(\mu_1, \Sigma_1)$, $\mathcal{P}_2=\mathcal{N}(\mu_2, \Sigma_2)$ есть многомерные нормальные распределения с невырожденными матрицами корреляции $\Sigma_{1,2}\in\mathbb{S}^n_{++}$, то дивергенция KL примет вид:
     
     $$2D_{KL}(\mu_1, \Sigma_1||\mu_2,\Sigma_2)=\text{tr}\left(\Sigma^{-1}_2\Sigma_1\right)-\log\det \left(\Sigma^{-1}_2\Sigma_1\right) + (\mu_1-\mu_0)^\top\Sigma^{-1}_2 (\mu_1-\mu_0) - n$$
     
     Hint: Пара трюков со следом: $\mathbf{a}^\top A\mathbf{a}=\text{tr}(\mathbf{a}^\top A\mathbf{a})=\text{tr}(\mathbf{a}\mathbf{a}^\top A);$ $\mathbb{E}[\text{tr}(AB)]=\text{tr}\left(\mathbb{E}[AB]\right).$
     \item $[2]$ Рассмотрим следующую задачу минимизации дивергенции Кульбака-Лейбера между нормальными распределениями при линейных ограничениях:
     
     \begin{equation*}
         \begin{aligned}
             &\min_{X\in\mathbb{S}^n_{++}} 2D_{KL}(\mathbf{0},X||\mathbf{0}, \Sigma)=\text{tr}\left(\Sigma^{-1}X\right)-\log\det \left(\Sigma^{-1}X\right)-n\\
               &\text{s.t. } X\mathbf{s}=\mathbf{y}
         \end{aligned}
     \end{equation*}
         где параметры имеют следующие размерности $A\in\mathbb{R}^{m\times n},\Sigma\in\mathbb{S}^n_{++}, \mathbf{s},\mathbf{y}\in\mathbb{R}^n$.Покажите, что оптимальное $X$ дается формулой:
    $$X=I+\mathbf{y}\mathbf{y}^\top -\frac{\mathbf{s}\mathbf{s}^\top}{\mathbf{s}^\top\mathbf{s}}=\left(I+\frac{\mathbf{y}\mathbf{s}^\top}{\|\mathbf{s}\|_2}-\frac{\mathbf{s}\mathbf{s}^\top}{\|\mathbf{s}\|_2}\right)\left(I+\frac{\mathbf{y}\mathbf{s}^\top}{\|\mathbf{s}\|_2^2}-\frac{\mathbf{s}\mathbf{s}^\top}{\|\mathbf{s}\|_2^2}\right)^\top$$
    в случае, если $\Sigma=I$ и $\mathbf{s}^\top \mathbf{y}=1$.
    
    \item $[1]$ Покажите, как свести задачу минимизации Кульбака-Лейбера с произвольной матрицей $\Sigma$ и векторами $\mathbf{s},\mathbf{y}$ такими, что $\mathbf{s}^\top\mathbf{y}>0$ к частному случаю, описанному в предыдущем случае. Выведите решение в этом случае, используя результат предыдущего пункта.
     \end{enumerate}

\end{enumerate}


\end{document}
